\documentclass{article}
\usepackage{amsmath, amssymb}
\usepackage[letterpaper, margin=1in]{geometry}
\usepackage{graphicx}
\usepackage{authblk}
\usepackage{libertine}
\usepackage[libertine]{newtxmath}

\title{Project 2 --- Face and Digit Classification}
\author{Fernando Gonzalez}
\author{Pranav Prakash}
\author{Wanyun Liu}
\affil{\textit {\{fdg17, pp618, wl432\}@scarletmail.rutgers.edu}}
\date{\today}

\begin{document}
  \maketitle
  \section{Classification Algorithms}
  \subsection{Naive Bayes}
  The Naive Bayes algorithm classifies images by keeping track of two sets of data.
  First are the {\em prior probabilities}, which are determined by:
  \begin{equation}
  \text{Prior}(y) = \Pr(Y = y) = \frac{\text{number of images with label = y}}{\text{total number of images}}
  \end{equation}
  Our goal in using the Naive Bayes classifier is to compute the probability that a given image has a certain label,
  given that a set of features is observed. To compute these conditional probabilities, we introduce Bayes' Rule:
  \begin{equation}
  \Pr(A \mid B) = \frac{\Pr(B \mid A)\Pr(A)}{\Pr(B)}
  \end{equation}
  Here, $A$ should refer to a {\em class}. The image classes are the set of all labels that can be given to an image.
  When classifying digits, these are the actual digits $\{0, 1, ..., 9\}$. For faces, the classes are "{\em not-face}" and "{\em face}",
  where "{\em not-face}" is represented by 0 or False, and "{\em face}" is represented by 1 or True.
  We let $Y, y$ refer to classes, and $X, x$ refer to features, where capitals are random variables. From (1), we have:
  \begin{equation}
  \Pr(Y = y \mid X_i = x_i) = \frac{\Pr(X_i = x_i \mid Y = y)\Pr(Y = y)}{\Pr(X_i = x_i)}
  \end{equation}
  Observing from (1) that $\Pr(Y = y)$ is the prior probability $\text{Prior}(y)$, we have:
  \begin{equation}
  \Pr(Y = y \mid X_i = x_i) = \frac{\Pr(X_i = x_i \mid Y = y)\text{Prior}(y)}{\Pr(X_i = x_i)}
  \end{equation}
  \subsection{Perceptron}
  \section{Implementation}
  \subsection{Features}
  \subsection{Results}
  \subsubsection{Naive Bayes}
  \subsubsection{Perceptron}
  \subsection{Obstacles}
\end{document}
